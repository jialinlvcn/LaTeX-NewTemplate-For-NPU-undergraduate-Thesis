%------------------------------------------------------------------------------%
%                                                                              %
%   LaTeX Template for Bachlor Thesis of Northwestern Polytechnical University %
%   Environment Config: TeXLive 2017                                           %
%   * XeTeX 3.14159265-2.6-0.99998 (TeX Live 2017/W32TeX)                      %
%   * BibTeX 0.99d (TeX Live 2017/W32TeX)                                      %
%   Version: 1.5.0.0426                                                        %
%                                                                              %
%------------------------------------------------------------------------------%
%   Copyright by NWPU Metaphysics Office, GPLv3-LICENSE                        %
%------------------------------------------------------------------------------%


%---------------------------------纸张大小设置---------------------------------%
\usepackage{geometry}
% 普通A4格式缩进
\geometry{left=1.25in,right=1.25in,top=1in,bottom=1in}
%------------------------------------------------------------------------------%


%----------------------------------必要库支持----------------------------------%
\usepackage{xcolor} %用于添加颜色支持,可以在文档中使用多种颜色。
\usepackage{tikz} %用于绘制图形和图表,支持各种绘图操作和样式。
\usepackage{layouts} %提供了一系列命令,用于显示和调整LaTeX文档的布局和尺寸。
\usepackage[numbers,sort&compress]{natbib} %用于控制参考文献的格式和样式,支持数字和作者年份两种引用风格。
\usepackage{clrscode} %提供了一些命令和环境,用于排版伪代码和算法描述。
\usepackage{gensymb} %提供了一些数学符号,如度数符号、温度符号等。
\usepackage[final]{pdfpages} %用于将外部PDF文件插入到LaTeX文档中,并控制其显示和排版方式。
%------------------------------------------------------------------------------%


%--------------------------------设置标题与目录--------------------------------%
\usepackage[sf]{titlesec} %将标题设置为无衬线字体
\usepackage{titletoc} %允许您更改目录的字体、格式、缩进和对齐方式等属性
%------------------------------------------------------------------------------%


%--------------------------------添加书签超链接--------------------------------%
\usepackage[unicode=true,colorlinks=false,pdfborder={0 0 0}]{hyperref}
% 在此处修改打开文件操作
\hypersetup{
    bookmarks=true,         % show bookmarks bar?
    pdftoolbar=true,        % show Acrobat’s toolbar?
    pdfmenubar=true,        % show Acrobat’s menu?
    pdffitwindow=true,      % window fit to page when opened
    pdfstartview={FitH},    % fits the width of the page to the window
    pdfnewwindow=true,      % links in new PDF window
}
% 在此处添加文章基础信息
\hypersetup{
    pdftitle={title},
    pdfauthor={author},
    pdfsubject={subject},
    pdfcreator={creator},
    pdfproducer={producer},
    pdfkeywords={key1  key2  key3}
}
%------------------------------------------------------------------------------%


%---------------------------------设置字体大小---------------------------------%
% 行距 1.25
% word 中的行距跟latex中的行距概念不一样。
% word 中,行距 = 单倍行距 x 行距值(1.25),而单倍行距 = 字体大小 x 固定系数(1.297/1.30)
% 默认字体12pt,所以 word 中行距 = 单倍行距 x 1.25 = 12pt x 1.3 x 1.25 = 19.5 pt
% latex 中,默认字体大小 12pt,可以通过设置其 baselineskip = 19.455 pt 来达到word要求的1.25倍行距的效果
% 又因为latex中默认 baselineskip = fontsize x 1.2,所以默认 baselineskip = 12pt x 1.2 = 14.4 pt
% 所以要求的 baselineskip 是默认的 baselineskip 的 19.455/14.4pt = 1.352 倍:
\usepackage{type1cm}
% 字号与行距,统一前缀s(a.k.a size)
%\linespread{1.354}
\newcommand{\sChuhao}{\fontsize{42pt}{63pt}\selectfont}                 % 初号, 1.5倍
\newcommand{\sYihao}{\fontsize{26pt}{36pt}\selectfont}                  % 一号, 1.4倍
\newcommand{\sErhao}{\fontsize{22pt}{28pt}\selectfont}                  % 二号, 1.25倍
\newcommand{\sXiaoer}{\fontsize{18pt}{18pt}\selectfont}                 % 小二, 单倍
\newcommand{\sSanhao}{\fontsize{16pt}{24pt}\selectfont}                 % 三号, 1.5倍
\newcommand{\sXiaosan}{\fontsize{15pt}{22pt}\selectfont}                % 小三, 1.5倍
\newcommand{\sSihao}{\fontsize{14pt}{21pt}\selectfont}                  % 四号, 1.5倍
\newcommand{\sHalfXiaosi}{\fontsize{12.5pt}{16.25pt}\selectfont}        % 半小四, 1.25倍
\newcommand{\sLargeHalfXiaosi}{\fontsize{13pt}{19pt}\selectfont}        % 半小四, 1.5倍
\newcommand{\sXiaosi}{\fontsize{12pt}{14.4pt}\selectfont}               % 小四, 1.25倍
\newcommand{\sLargeWuhao}{\fontsize{11pt}{11pt}\selectfont}             % 大五, 单倍
\newcommand{\sWuhao}{\fontsize{10.5pt}{10.5pt}\selectfont}              % 五号, 单倍
\newcommand{\sXiaowu}{\fontsize{9pt}{9pt}\selectfont}                   % 小五, 单倍
%------------------------------------------------------------------------------%


%---------------------------------设置中文字体---------------------------------%
\usepackage{fontspec}
\usepackage[SlantFont,BoldFont,CJKchecksingle]{xeCJK}
\usepackage{CJKnumb}
% 使用 Adobe 字体
\newcommand\adobeSog{SimSun}
\newcommand\adobeHei{SimHei}
\newcommand\adobeKai{KaiTi}
\newcommand\adobeFag{FangSong}
\newcommand\codeFont{Consolas}

% 设置字体
\defaultfontfeatures{Mapping=tex-text}
\setCJKmainfont[ItalicFont=\adobeKai, BoldFont=\adobeHei]{\adobeSog}
\setCJKsansfont[ItalicFont=\adobeKai, BoldFont=\adobeHei]{\adobeSog}
\setCJKmonofont{\codeFont}
\setmonofont{Times New Roman}
% 设置字体族
\setCJKfamilyfont{song}{\adobeSog}      % 宋体
\setCJKfamilyfont{hei}{\adobeHei}       % 黑体
\setCJKfamilyfont{kai}{\adobeKai}       % 楷体
\setCJKfamilyfont{fang}{\adobeFag}      % 仿宋体
% 用于页眉学校名,特殊字体,powerby https://github.com/ecomfe/fonteditor
\setCJKfamilyfont{nwpu}{nwpuname}
% 新建字体命令,统一前缀f(a.k.a font)
\newcommand{\fSong}{\CJKfamily{song}}
\newcommand{\fHei}{\CJKfamily{hei}}
\newcommand{\fFang}{\CJKfamily{fang}}
\newcommand{\fKai}{\CJKfamily{kai}}
\newCJKfontfamily{\fNWPU}[Path=settings/]{nwpuname.ttf}
%------------------------------------------------------------------------------%


%------------------------------添加插图与表格控制------------------------------%
\usepackage{graphicx}
\usepackage[font=small,labelsep=quad]{caption}
\usepackage{wrapfig}
\usepackage{multirow,makecell}
\usepackage{longtable}
\usepackage{booktabs}
\usepackage{tabularx}
\usepackage{setspace}
\captionsetup[table]{labelfont=bf,textfont=bf}
%------------------------------------------------------------------------------%


%---------------------------------添加列表控制---------------------------------%
\usepackage{enumerate}
\usepackage{enumitem}
%------------------------------------------------------------------------------%


%---------------------------------设置引用格式---------------------------------%
\renewcommand\figureautorefname{图}
\renewcommand\tableautorefname{表}
\renewcommand\equationautorefname{式}
\newcommand\myreference[1]{[\ref{#1}]}
\newcommand\eqrefe[1]{式(\ref{#1})}
% 增加 \ucite 命令使显示的引用为上标形式
\newcommand{\ucite}[1]{$^{\mbox{\scriptsize \cite{#1}}}$}
\renewcommand\arraystretch{1.4}
\renewcommand\theequation{\thesection.\arabic{equation}}
\renewcommand{\thefigure}{\thechapter-\arabic{figure}}
\renewcommand{\thetable}{\thechapter-\arabic{table}}
%将\listoffigures和\listoftables的标题改为中文,可以在附录中直接使用这两个命令生成图片列表和表格列表
\renewcommand{\listfigurename}{插图列表}
\renewcommand{\listtablename}{表格列表}

\usepackage{float}
\setlength{\textfloatsep}{25pt plus 1.0pt minus 2.0pt}
\setlength{\floatsep}{25pt plus 1.0pt minus 2.0pt}
\setlength{\intextsep}{25pt plus 1.0pt minus 2.0pt}

\setlength{\bibsep}{0em} % 设置参考文献条目间的距离
%------------------------------------------------------------------------------%


%--------------------------------设置定理类环境--------------------------------%
\usepackage[amsthm,thmmarks]{ntheorem}
\newtheorem{myexample}{例}
\newtheorem{thm}{定理}
%------------------------------------------------------------------------------%


%--------------------------设置中文段落缩进与正文版式--------------------------%
\XeTeXlinebreaklocale "zh"                      % 使用中文的换行风格
\XeTeXlinebreakskip = 0pt plus 1pt              % 调整换行逻辑的弹性大小
\usepackage{indentfirst}                        % 段首空格设置
\setlength{\parindent}{26pt}                    % 段首空格长度
\setlength{\parskip}{3pt plus 1pt minus 1pt}    % 段落间距
%根据word和latex的行间距转化公式 1.25*1.3/1.2
\renewcommand{\baselinestretch}{1.354}           % 行距
%------------------------------------------------------------------------------%

%---------------------------------设置页眉页脚---------------------------------%
\usepackage{fancyhdr}
\usepackage{fancyref}
%\addtolength{\headsep}{-0.1cm}          %页眉位置
\addtolength{\footskip}{-0.1cm}         %页脚位置
\addtolength{\topmargin}{0cm}
\newcommand{\makeheadrule}{
	\makebox[0pt][l]{\rule[.7\baselineskip]{\headwidth}{0.8pt}}
	\vskip-.8\baselineskip
}
\makeatletter
\renewcommand{\headrule}{%
	{\if@fancyplain\let\headrulewidth\plainheadrulewidth\fi\makeheadrule}}
\pagestyle{fancyplain}
\fancyhf{}
\fancyfoot[C,C]{\sXiaowu~\thepage~} % 设置页脚字体大小
% 后续文字可以自行修改
\chead{\sSanhao\raisebox{0.04cm}%
	{ \fNWPU 西北工业大学} \fSong{{\textbf{本科毕业设计(论文)} }}}
%------------------------------------------------------------------------------%

%----------------------------设置段落标题与目录格式----------------------------%
\setcounter{secnumdepth}{3}
\setcounter{tocdepth}{2}

\newcommand\chapterID[1]{第\CJKnumber{#1}章}
\renewcommand{\chaptername}{第~\CJKnumber{\thechapter}~章}
\renewcommand{\figurename}{图}
\renewcommand{\tablename}{表}
\renewcommand{\bibname}{参考文献}
\renewcommand{\contentsname}{\bfseries 目~录}
\newcommand{\keywords}[1]{\\ \\ \textbf{关键词}:#1}
%\renewcommand{preface/c_abstract}{中文摘要(脑机接口,机器学习,深度学习)}
%\renewcommand{preface/c_abstract}{英文摘要(Brain-Computer\enspace Interface,\enspace Machine\enspace Learning,\enspace Deep\enspace Learning)}



\titleformat{\chapter}[hang]{\normalfont\sSanhao\filcenter\fHei\bf}%
{\sSanhao{\chaptertitlename}}{1em}{\sSanhao}
\titleformat{\section}[hang]{\fHei \bf \sXiaosan}%
{\sXiaosan \thesection}{0.5em}{}{}
\titleformat{\subsection}[hang]{\fHei \bf \sLargeHalfXiaosi}%
{\sLargeHalfXiaosi \thesubsection}{0.5em}{}{}
\titleformat{\subsubsection}[hang]{\fHei \bf}%
{(\arabic{subsubsection})}{0.5em}{}{}   % 小标题式的subsubsection:(4) 标题  % 小标题式的subsubsection:(4) 标题

% 缩小正文中各级标题之间的缩进
\titlespacing{\chapter}{0pt}{15.2pt}{14.4pt}
\titlespacing{\section}{0pt}{-3pt}{2pt}
\titlespacing{\subsection}{0pt}{-2.5pt}{-2.5pt}
\titlespacing{\subsubsection}{0pt}{0.25em}{0pt}

% 定义目录中各级标题之间的格式以及缩进
\dottedcontents{section}[1.16cm]{}{1.8em}{5pt}
\dottedcontents{subsection}[2.00cm]{}{2.7em}{5pt}
\dottedcontents{subsubsection}[2.86cm]{}{3.4em}{5pt}
\titlecontents{chapter}[0pt]{\fHei\vspace{0.5em}}%
{\contentsmargin{0pt}\fHei\makebox[0pt][l]{\chapterID{\thecontentslabel}}\hspace{3.8em}}%
{\contentsmargin{0pt}\fHei}%
{\titlerule*[.5pc]{.}\contentspage}[\vspace{0em}]
%------------------------------------------------------------------------------%





%----------------------------------其他补充设置--------------------------------%
% 重置列表环境的间隔
% \let\orig@Itemize =\itemize
% \let\orig@Enumerate =\enumerate
% \let\orig@Description =\description

% \def\Myspacing{
%     \itemsep=1.5ex \topsep=-0.5ex \partopsep=0pt \parskip=0pt \parsep=0.5ex
% }

% \def\newitemsep{
%     \renewenvironment{itemize}{\orig@Itemize\Myspacing}{\endlist}
%     \renewenvironment{enumerate}{\orig@Enumerate\Myspacing}{\endlist}
%     \renewenvironment{description}{\orig@Description\Myspacing}{\endlist}
% }

% \def\olditemsep{
%     \renewenvironment{itemize}{\orig@Itemize}{\endlist}
%     \renewenvironment{enumerate}{\orig@Enumerate}{\endlist}
%     \renewenvironment{description}{\orig@Description}{\endlist}
% }

% \newitemsep
% 下划线
\newcommand\dlmu@underline[2][5cm]%
{\hskip1pt\underline{\hb@xt@ #1{\hss#2\hss}}\hskip3pt}
\let\coverunderline\dlmu@underline
%------------------------------------------------------------------------------%


%----------------------------------添加代码控制--------------------------------%
\usepackage{listings}
\lstset{
    basicstyle=\footnotesize\ttfamily,
    numbers=left,
    numberstyle=\tiny,
    numbersep=5pt,
    tabsize=4,
    extendedchars=true,
    breaklines=true,
    keywordstyle=\color{blue}\bfseries,
    numberstyle=\color{purple},
    commentstyle=\color[rgb]{0, 0.4, 0}\bfseries,
    stringstyle=\color{violet}\ttfamily\bfseries,
    rulesepcolor=\color{red!20!green!20!blue!20},
    showspaces=false,
    showtabs=false,
    frame=shadowbox,
    framexrightmargin=5pt,
    framexbottommargin=4pt,
    showstringspaces=false,
    escapeinside=`', %逃逸字符(1左面的键),用于显示中文
}
\renewcommand{\lstlistingname}{CODE}
\lstloadlanguages{% Check Dokumentation for further languages, page 12
    Pascal, C++, Java, Ruby, Python, Matlab, R, Haskell
}
%------------------------------------------------------------------------------%

\endinput
% 这是简单的 thesis(book) 的导言区设置,不能单独编译。
